%vi: ft=tex
% Chapter 1

\chapter{Multivariable Differential Analysis} % Main chapter title

\label{Chapter1} % For referencing the chapter elsewhere, use \ref{Chapter1} 
\lhead{Chapter 1. \emph{MVC}} % This is for the header on each page - perhaps a shortened title


\section{Introduction and Overview}
Single variable calculus was a system of shortcuts formulated in the 17th century,
to present in a simple manner the results obtained by the method of exhaustion, a
method used since the time of Euclid to calculate areas and volumes of figures
and surfaces. Along with Descartes' Analytic Geometry, existent algebra and
Newton's calculus of infinite series, a complete system for integration and
differentiation of all functions of single variable expressible as power series
was possible.\cite{stillwell}
\newline
Multivariable calculus is the next natural step in the process, applying the
same techniques and principles from single variable calculus to functions of
multiple variables.
\newline
The rigourous analysis and verification of the tools and techniques of calculus
to place it on a logical and deductive footing constitutes Mathematical Analysis. Hence,
multivariable differential analysis is the branch concerned with proofs of
validity of techniques used in multivariable differential calculus.
\newline
Contents in this chapter are majorly guided by the study of books
Apostol\cite{apostol}, and Stillwell\cite{stillwell}.

\section{Directorial Derivative}
Let U be any subset of $\bm{R}^n$ and let $\bm{f} : U \rightarrow \bm{R}^m$  be a function.
Let $\bm{u}$ be any direction in $\bm{R}^m$ and $\bm{c}$ be any point in domain,
then the directorial derivative aims to estimate the change in $\bm{f}$ as it
moves from point $\bm{c}$ to neighbouring points along the direction $\bm{u}$.
\begin{definition}
    Directorial derivative of $\bm{f}$ at point $\bm{c}$ in direction $\bm{u}$ is
    denoted by $\bm{f(c;u)}$ and defined as 
    \[ \bm{f(c;u)} = \lim_{h \to 0}
    \frac{\bm{f}(\bm{c}+h\bm{u})-\bm{f(c)}}{h} \]
\end{definition}
\begin{lemma}
    Existence of all directorial derivatives at a point is not a sufficient
    condition for continuity of the function.
\end{lemma}
%\begin{proof}
%Consider the counter example,
%TODO
%\end{proof}
\begin{corollary}
    Directorial derivatives may or may not be linear operators.
\end{corollary}


\section{Total Derivative}
The directorial derivative concept, although useful, has proven insufficient as
an analog to single variable derivative becuase of two primary reasons, as shown
above : lack of failing to preserve continuity and the possibility of
non-linearity. So a better alternative is needed and defined as follows.
\begin{definition}
    The total derivative of a function  $\bm{f} : \bm{U} \rightarrow \bm{R}^m$ at point
    $\bm{c}$, if it exists, is a linear operator $\bm{A_{c}} : \bm{R}^n \rightarrow \bm{R}^m$ such
    that
    \[
        \bm{f(c+v) = f(c) + A_{c}(v) + ||v||E_{c}(v)}
    \]
    where $\bm{E_{c}(v) \rightarrow 0}$ as $\bm{v\rightarrow 0}$, $\forall \bm{v}$
    $\epsilon$ $\bm{U}$. Here $\bm{U}$ is an n-ball centered at $\bm{c}$.
\end{definition}

\subsection{Properties of Total Derivative}
\begin{theorem}
    If the total derivative of a function $\bm{f} : \bm{U} \rightarrow \bm{R}^m$
    exists at point $\bm{c}$, then
    \[
        \bm{A_{c}(v) = f(c;v)}.
    \]
    Here $\bm{U}$ is an n-ball centered at $\bm{c}$.
\end{theorem}
\begin{proof}
    Let the total derivative of a function $\bm{f}$ exist in the n-ball
    $B(\bm{c}, r)$, and let $\bm{v} = h\bm{u}$
    \[
        \bm{f(c+\text{h}u) = f(c) + \text{h}A_c(u) + ||\text{h}u||E(\text{h}u)}
    \]
    Rearranging terms,
    \[
        \bm{f(c+\text{h}u) - f(c) = \text{h}A_c(u) + ||\text{h}u||E(\text{h}u)}
    \]
    On dividing by h and taking the limit to 0, error term goes to 0 and
    $\bm{A_c(u) = f(c;u)}$ by definition.
\end{proof}
\begin{corollary}
    If the total derivative for a function exists at a point, then it is unique.
\end{corollary}
\begin{proof}
    The directorial derivative, by definition, is a unique limit. By above
    theorem, whenever the total derivative exists, it'll be unique.
\end{proof}

\begin{theorem}
    If a function $\bm{f} : \bm{U} \subseteq \bm{R}^n \rightarrow \bm{R}^m$ is
    differentiable at a point $\bm{c}$, then
    \[
        \bm{A_c(v) = f(c;v) = \sum_{i = 1}^{n} v_i D_i f(c)}
    \]
    where, $\bm{v} = v_1\bm{e_1} +...+ v_n\bm{e_n}$ and $\bm{D_kf(c)}$ is the
    kth partial derivative of function $\bm{f}$ at point $\bm{c}$.
\end{theorem}
\begin{proof}
    Since the total derivative exists, directorial derivative exists and is linear
    in $\bm{v}$ at point $\bm{c}$.
    \[
        \bm{f(c;v) = f(c;\sum_{i=1}^{n} v_i e_i)}
    \]
    \[
        \bm{f(c;v) = \sum_{i=1}^{n} v_i f(c;e_i)}
    \]
    \[
        \bm{f(c;v) = \sum_{i=1}^{n} v_i D_if(c)}
    \]
\end{proof}
\begin{theorem}
    If a function $\bm{f} : \bm{U} \subseteq \bm{R}^n \rightarrow \bm{R}^m$ is
    differentiable at a point $\bm{c} \epsilon \bm{U}$, then it is continuous 
    at that point.
\end{theorem}
%\begin{proof}
%TODO
%\end{proof}

Following theorem derives the matrix form of the linear operator $\bm{A_c(v)}$
\begin{theorem}
    Let $\bm{f} : \bm{U} \subseteq \bm{R}^n \rightarrow \bm{R}^m$ be a function
    with derivative defined at point $\bm{c}$ in $\bm{U}$. Let
    $\{\bm{g_i}\}_{i=1}^{n}$ be the standard basis for $\bm{R}^n$, and let
    $\{\bm{e_i}\}_{i=1}^{m}$ be the standard basis for $\bm{R}^m$. Then the
    matrix of linear transformation for the derivative defined w.r.t to these
    basis will be
    \[
        \bm{A_{c}} =
        \begin{pmatrix}
            D_{1}f_{1} & D_{2}f_{1} & \cdots & D_{n}f_{1} \\
            D_{1}f_{2} & D_{2}f_{2} & \cdots & D_{n}f_{2} \\
            \vdots  & \vdots  & \ddots & \vdots  \\
            D_{1}f_{m} & D_{2}f_{m} & \cdots & D_{n}f_{m}
        \end{pmatrix}
    \]
    where,
    \[
        \bm{f(v)} = \sum_{i=1}^{m} f_i(\bm{v}) \bm{e_i}
    \]
    and $f_i : \bm{U} \subseteq \bm{R}^n \rightarrow \bm{R}$.
\end{theorem}
\begin{proof}
    \[
        \bm{A_c(v) = \sum_{i = 1}^{n} v_i D_i f(c)}
    \]
    \[
        \bm{A_c(g_i) = \sum_{i = 1}^{n} v_i D_i f(g_i)}
    \]
    \[
        \bm{A_c(g_i) = D_i f(g_i)}
    \]
    \[
        \bm{A_c(g_i)} = D_i \sum_{j=1}^{n}f_j(g_i)\bm{e_j}
    \]
    \[
        \bm{A_c(g_i)} = \sum_{j=1}^{n} D_i f_j(g_i)\bm{e_j}
    \]
    Representing in matrix form using the theorems in linear algebra, we get the
    desired matrix.
\end{proof}
\begin{definition}
    The matrix form of total derivative of a function $\bm{f} : \bm{U} \subseteq
    \bm{R}^n \rightarrow \bm{R}^m$ at point $\bm{c}$ is called the Jacobian
    matrix, as described in the above theorem.
\end{definition}

%TODO Chain rule, sufficient condition, equality of mixed partial derivatives,
%Taylor's theorem for successive approximations
