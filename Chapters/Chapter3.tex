% vi: filetype=tex
% Chapter 3
\chapter{Measures} % Main chapter title

\label{Chapter3} % For referencing the chapter elsewhere, use \ref{Chapter3} 

\lhead{Chapter 3. \emph{Measures}} % This is for the header on each page - perhaps a shortened title


\section{Introduction and Overview}
The theory of measures, in its current form, developed along with the theory of
sets.
The principle concern was to develop a rigourous theory of integration of
functions, which at the time was widely used to caculate areas and volumes,
decompose functions into Fourier Series etc.
<fourier series>
\newline
The first rigourous theory of integration of discontinuous functions, and
continuous functions was developed by Riemann and was based on step-functions.
The theory successfully deals with continuous functions having a finite number
of discontinuities, and some functions with infinite number of discontinuities
but is unadequate to tackle some others like the Dirichlet function.
<Dirichlet Function>
\newline
As the events unfolded, it became clearer that the crucial part which was
missing in the rigourous theory of integration was a rigourous theory of
`measurement' of arbitrary sets.
\newline
%\subsection{Mathematical Developments and Measure Theory in Brie}
The determining event, as recognised, leading to a need for a better theory of
measurement was Harnack's result that any countable set can be covered by an
interval or arbitrary length, in $\bm{R}$. This was counter-intuitive for dense
sets in $\bm{R}$ like $\bm{Q}$.
\newline
The primary response to this, by Jordan, consisted of a Riemann analogue and
used finite unions of intervals to as an approximation to arbitrary set in
$\bm{R}$.
According to this measure, the sparse sets did have measure zero but dense sets
weren't measurable at all.
\newline
Next came using countable union of intervals to define measure as we do now, and
approximate sets in $\bm{R}$. Borel's idea was then `completed' by Lebesgue to
include all subsets of null sets(as explained later).
\newline
Lastly, the development of theory of measures has had huge impact on the
development of the foundational theory of sets. Conflict arose between options
like 'The Axioms of Choice' and the powerset of R to be a measurable space, and
the choosing of the former lead to existence of some non-measurable sets in
$\bm{R}$.

\subsection{Desired Properties of Measure}
The properties desired from the concept which assigns a numerical value to the
content of sets(in $\bm{R}^n$) are,
\begin{enumerate}
    \item Well-defined : Well definition means that the measure of any set
        should be independent of the way it is calculated, for the set. For
        example, the area of the triangle in a 2-D Euclidean plane should be
        independent of the method of calculuation.
    \item Invariant under congruences : This means that the measure for
        congruent sets in $\bm{R}^n$ should be the same. Another way to express
        it is that the measure should be invariant under linear translations of
        sets in $\bm{R}^n$.
    \item Countable-additivity or $\sigma$-additivity : This means that 
        \begin{equation}
            measure(\bigcup_{i=1}^{\infty} F_{i}) = \bigcup_{i=1}^{\infty}
            measure(F_{i}) 
        \end{equation}
        for \textbf{mutually disjoint} sets $F_{i}$.
        The reason for including countable additivity, in addition to finite
        additivity for disjoint sets, is to account for the stability of the
        limiting cases. For example, the addition of areas of rectangles as done
        in Riemann Integrals and incorporating the limiting concept.
\end{enumerate}


\section{$\sigma$-Algebras and Set-Theoretic Properties}
Out of the need to satisfy countable-additivity, the measure function is to be
defined the sets termed $\sigma$-algebras and for the logical stability under
basic set theoretic operations.
\begin{definition}
    A $\sigma$-algebra A defined on X is a subset of P(X), having the following
    properties,
    \[
        X,\phi\:\epsilon\:A
    \]
    \[
        S\:\epsilon\:A \implies A^c \: \epsilon \: A
    \]    
    \[
        (S_i)_{i\epsilon\bm{N}} \implies \bigcup_{i=1}^{\infty} S_{i} \:
        \epsilon \: A
    \]
\end{definition}

Examples
Properties
\begin{theorem}
    The intersection of arbitrarily many $\sigma$-algebras in X is also a
    $\sigma$-algebra in X.
\end{theorem}
\begin{proof}
    Let $(A_i)_{i \epsilon I}$ be a collection $\sigma$-algebras, and let
    \[
        \bigcap_{i \: \epsilon \: I}A_i = A
    \]
    then,
    \begin{enumerate}
        \item X, $\phi \: \epsilon$ A, because X, $\phi \epsilon A_i \: \forall
        \: i$
        \item Let B $\epsilon$ A, then $B^c \: \epsilon$ A because B belongs to
            all $A_i$ and hence $B^c$ belongs to all $A_i$
        \item Same argument as above.
    \end{enumerate}
\end{proof}
\begin{definition}
    Let G $\subseteq$ P(X), then the smallest $\sigma$-algebra containing G is
    termed $\sigma$(G) and the set G is called its generator.
\end{definition}
\begin{theorem}
    For any G $\subseteq$ P(X), $\sigma$(G) exists.
\end{theorem}
\begin{proof}
    Let 
    \[
        \sigma(G) = \bigcap_{G \subseteq A} A_i
    \]
    , where $A_i$ are $\sigma$-algebras which have G as a subset. By previous
    theorem, $\sigma$(G) is also a $\sigma$-algebra. Also, $\sigma$(G) is
    non-empty because P(X) is one $\sigma$-algebra s.t $G \subseteq P(X)$.
\end{proof}
%DOUBT in proof : How is indexing done for such arbitrary sets???

\subsection{Borel $\sigma$-Algebra or Borel Sets}
\begin{definition}
    Let $O^n$ be the set of open sets in $\bm{R}^n$, and  $\sigma$-algebra
    generated by $O^n$ $\sigma(O^n)$, then the members of $\sigma(O^n)$ are
    called borel sets or borel-measurable sets. They are also denoted by
    $B(\bm{R}^n)$.
\end{definition}
\begin{theorem}
    Let $O^n$, $C^n$ and $K^n$ denote open, closed and compact sets in
    $\bm{R}^n$ respectively, then
    \[
        B(\bm{R}^n) \: = \sigma(O^n) \: = \: \sigma(C^n) \: = \: \sigma(K^n)
    \]
\end{theorem}
\begin{proof}
\end{proof}

\section{Measures}
\begin{definition}
    Let X be any set and let A be a $\sigma$-algebra on it, $\mu \: : \:
    A \rightarrow [0,\infty]$ is called a measure if it follows ths following
    properties,
    \[
        \mu(\phi) \: = \: 0
    \]
    \[
        \mu(\bigcup_{i=1}^{\infty} D_{i}) = \bigcup_{i=1}^{\infty}
        \mu(D_{i})
    \]
    The second property is termed $\sigma$-additivity.
    (X, A, $\mu$) is called a measure space.
\end{definition}
Examples of measures
\begin{lemma}
    Let (X, A, $\mu$) be a measure space and B, C $\epsilon$ A, then
    \begin{enumerate}
        \item B $\subset$ C $\implies$ $\mu(B) \: \leqslant \: \mu(C)$
        \item B $\subset$ C $\implies$ $\mu(B \setminus C) \: = \: \mu(B) \: - \:
            \mu(C)$
        \item $\mu(B \cup C) \leqslant \mu(B) + \mu(C)$
    \end{enumerate}
\end{lemma}
Of course, easily verified by decomposition into disjoint sets.
\begin{definition}
    An exhausting sequence of sets in X is an increasing sequence such that the
    limit of the sequence is the space X itself.
\end{definition}
\begin{theorem}
    Let (X, A) be a measurable space. $\mu \: : \: A \rightarrow [0,\infty]$ is
    a measure iff
    \begin{enumerate}
        \item $\mu(\phi) = 0$
        \item $\mu(B \cup C) = \mu(B) + \mu(C)$, for disjoint sets B and C in A
        \item let $(B_i)_{i \epsilon \bm{N}}$ be any increasing sequence such
            that $B_i \: \uparrow \: B$, then
            \[
                \mu(B) = \lim_{i \rightarrow \infty} \mu(B_i)
            \]
    \end{enumerate}
\end{theorem}
\begin{proof}
\end{proof}
\begin{theorem}
    Every measure (or pre-measure) is follows $\sigma$-subadditivity,
    \[
        \mu(\bigcup_{i \epsilon \bm{N}}(B_i) \leqslant \sum_{i \epsilon N}\mu(B_i)
    \]
\end{theorem}
\begin{proof}
\end{proof}
\begin{definition}
    Let (X, A, $\mu$) be a measure space, N $\epsilon$ A is called a null set if
    $\mu$(N) = 0. The set of all null sets for a measure space is called its
    null space and denoted by $N_{\mu}$.
\end{definition}
\begin{lemma}
    Null space of a measure space has the following properties,
    \begin{enumerate}
        \item $\phi \: \epsilon \: N_{\mu}$
        \item $N \: \epsilon \: N_{\mu}, \: M \subset \: N \implies \mu(M) \: = \:
            0$, here both M and N $\epsilon$ A.
        \item $(N_j)_{j\epsilon\bm{N}} \: \subset \: N_\mu \implies
            \bigcup_{j\epsilon\bm{N}} \: \epsilon \: N_\mu$
    \end{enumerate}
\end{lemma}
\begin{proof}
    \begin{enumerate}
        \item By definition of measure, $\mu(\phi)$ = 0.
        \item $\mu(X)$ = $\mu(N)$ + $\mu(N^c)$ = $\mu(M)$ + $\mu(M^c)$
            \newline 0 + $\mu(N^c)$ = $\mu(M)$ + $\mu(N^c)$ + $\mu(M^c \:
            \setminus \: N^c)$
            \newline 0 = $\mu(M)$ +  + $\mu(M^c \: \setminus \: N^c)$
            \newline Since both quantities on R.H.S are greater than or equal
            to 0, both are 0
        \item Directly follows from $\sigma$-sub-additivity.

    \end{enumerate}
\end{proof}
\begin{definition}
    A measure space (X, $A^*$, $\mu$) is called complete if all the subsets of
    every null set also belong to $A^*$. Equivalently; if all the subsets of any
    null set are also null sets and hence have 0 measure and belong to the
    $\sigma$-algebra $A^*$, then a measure space is called complete.
\end{definition}
\begin{theorem}
    Any incomplete measure space can be completed.
\end{theorem}
\begin{proof}
    \begin{enumerate}
        \item Define $A^*$ = \{ S $\cup$ N; S $\epsilon$ A, N $\subset$ M
            $\epsilon \: N_\mu$ \}
    \end{enumerate}
\end{proof}

\section{Important Measures}
\subsection{Lebesgue Measure}
\begin{definition}
    Let [[a,b)) = $[a_1,b_1) \: \times...\times \: [a_n,b_n)$ be a half- open
    rectangle $\epsilon$ $J^n$, then
    \[
        \lambda^n([[a,b))) \: = \: \prod_{i=1}^{\infty}(b_i - a_i)
    \]
        defines the set function $\lambda^n \: : \: J^n \rightarrow \bm{R}$
        called the n-dimensional Lebesgue measure.
\end{definition}
Important results concerning Lebesgue measure are not provided due to standard
familiarity with the topic.

%TODO
%\subsubsection{Note on Non-Measurable Sets}
%In this section, the results are presented in the spirit of a complete picture,
%but not proved.


\section{Hausdorff Measure}
Hausdorff measure even further generalises the notion of Lebesgue measure,
\begin{definition}
    The diameter of a set S in a metric space is defined as
    \[
        |S| = sup\{|x-y| \: : x,y \:  \: \epsilon \: S \}
    \]
    .
\end{definition}
\begin{definition}
    A $\delta$-cover of a set F is an open cover of F such that every set in
    $\{U_i\}$ has diameter at most $\delta$.
\end{definition}
Let for any set F in a metric space
\[
    H_\delta^s(F) = inf\{|U_i|^s : \{U_i\} \text{ is a $\delta$-cover of F} \}
\]
\begin{definition}
    s-dimensional Hausdorff measure of a set F in a metric space (X, d) is 
    defined as
    \[
        H^s(F) = \lim_{\delta \to 0} H_\delta^s(F)
    \]
    $\forall F \subseteq X$.
\end{definition}
\begin{theorem}
    Hausdorff measure on a metric space is an outer measure.
\end{theorem}
\begin{theorem}\label{thm_haus_mea_lim}
    Let $F \subseteq X$, where (X, d) is a metric space
    \begin{enumerate}
        \item \[
                0 < \delta < \gamma \implies H_\delta^s (F) \geqslant
                H_\delta^s (F) 
            \]
        \item \[
                H^s (F) = \sup_{\delta > 0} H_\delta^s(F)
            \]
        \item If $0 < s < t$ and $0 < \delta < $, then
            \[
                H_\delta^t(F) \leqslant H_\delta^t(F)
            \]
    \end{enumerate}
\end{theorem}
\begin{theorem}
    The s-dimensional Hausdorff measure has the following properties in
    $\bm{R}^n$
    \begin{enumerate}
        \item Generalisation of n-dimensional volume : Let $F \epsilon
            B(\bm{R}^n)$, then
            \[
                H^n(F) \propto \mu_n(F)
            \]
            , where $\mu_n : \bm{B^*(R^n)} \to [0, \infty]$ is the n-dimensional lebesgue
            measure.
        \item For lower-dimensional subset in $\bm{R}^n$ : $H^0(F)$ gives number
            of points if finite, and infinity otherwise
            \newline similarly, $H^1(F)$ gives the length of a smooth curve
            \newline and so on
    \end{enumerate}
\end{theorem}

%TODO : show examples for the hausdorff measure, and do them yourselves.
